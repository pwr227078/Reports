\documentclass[12pt, a4paper, oneside]{article}
\usepackage{graphicx}
\usepackage{arial}
\renewcommand{\familydefault}{\sfdefault}
\usepackage[T1]{fontenc}
\usepackage[polish]{babel}
\usepackage[utf8]{inputenc}
\usepackage{lmodern}
\usepackage[left=2cm,right=2cm,top=2cm,bottom=2cm]{geometry}
\selectlanguage{polish}

\begin{document}
\pagenumbering{arabic}
\section{Wstęp}
W DPCM - Differential Pulse Coded Modulation, przy niewielkiej szybkości zmian amplitudy sygnału źródłowego, rozważamy kodowanie jedynie przyrostów pomiędzy próbkami. Pozwala to uniknąć kodowania nadmiernej informacji w postaci składowej stałej. Prowadzi to do redukcji bitowej reprezentacji. Podstawowym przykładem realizacji DPCM jest Modulacja Delta (DM). W klasycznej DM problem stanowi jednak dobranie optymalnej wartości $\Delta$.
\subsection{ADM - Adaptive Delta Modulation}
Problem ten został rozwiązany w ADM poprzez adaptacyjną zmianę parametru $\Delta$. Istnieje wiele reguł aktualizacji delty, jednak najpopularniejszą z nich jest poniższa formuła:
\begin{equation}
\Delta(n)=\Delta(n-1)\cdot K^{e_{wy}(n)\cdot e_{wy}(n-1)},~\Delta(0)\geq0,~K>1
\end{equation}
W przypadku zbyt szybko narastającego lub zbyt szybko malejącego zbocza $\Delta$ zostaje zwiększona, a~przy sygnale wolnozmiennym - zmniejszona. Pozwala to na ciągłe dopasowywanie się do sygnału. Przy zbyt dużych wartościach parametru K może dojść do powstania szumu śrutowego. Odpowiedni dobór $\Delta$(0) oraz K jest tematem tego laboratorium.
\subsection{ADPCM - Adaptive Differential Pulse Coded Modulation}
W celu zwiększenia efektywności działania układu ADM zaproponowano wykorzystanie filtra adaptacyjnego zbudowanego na strukturze stabilnej w sensie BIBO - filtr FIR. W praktyce kodery ADPCM to predykcyjne filtry cyfrowe składające z~części statycznej - filtra IIR oraz adaptacyjnej - filtra FIR. W~uproszczonej analizie opisywany jest jedynie część adaptacyjna. ADPCM cechuje się następującą złożonością obliczeniową:
\begin{equation}
R=f_s(3p+1)=f_s\cdot K,
\end{equation}
gdzie:
\begin{itemize}
\item R - ilość mnożeń wykonywanych w ciągu sekundy,
\item K - ilość mnożeń elementarnych,
\item f$_s$ - częstotliwość próbkowania,
\item p - długość filtru.
\end{itemize}
\begin{figure}[h]
\centering
\caption{Schemat kodera ADPCM z 4 - bitowym kwantyzerem dynamicznym}
\includegraphics[scale=0.46]{f1.png}
\end{figure}
\clearpage
\section{Wyznaczanie optymalnych wartości parametrów $\Delta$ oraz K}
\subsection{Wyznaczenie optymalnej wartości $\Delta$(0)}
\begin{figure}[h]
\centering
\caption{Wykres $\Delta$(n) dla K = 1.01}
\includegraphics[scale=0.33]{f1.jpg}
\end{figure}
\begin{center}
Okres przejściowy T$_p$ = 600 próbek
\end{center}
\begin{figure}[h]
\centering
\caption{Wykres $\Delta$(n) dla K = 1.1}
\includegraphics[scale=0.33]{f2.jpg}
\end{figure}
\begin{center}
Okres przejściowy T$_p$ = 160 próbek
\end{center}
\clearpage
\begin{figure}[h]
\centering
\caption{Wykres $\Delta$(n) dla K = 1.5}
\includegraphics[scale=0.33]{f3.jpg}
\end{figure}
\begin{center}
Okres przejściowy T$_p$ = 145 próbek
\end{center}
\begin{figure}[h]
\centering
\caption{Wykres $\Delta$(n) dla K = 2.0}
\includegraphics[scale=0.33]{f4.jpg}
\end{figure}
\begin{center}
Okres przejściowy T$_p$ = 100 próbek
\end{center}
\indent\indent Z powyższych wykresów wynika, że wraz ze wzrostem wartości K maleje okres przejściowy. Jednak dla dowolnej wartości parametru K, jeśli przyjąć $\Delta$(0) = 0.01  to okres przejściowy T$_p$ = 0. Do dalszych obliczeń (po wyznaczeniu obszaru pracy) przyjęta zostanie więc $\Delta$(0) = 0.01. Redukuje to problem optymalizacji funkcji kosztu do problemu 1D.
\clearpage
\subsection{Wyznaczanie obszaru pracy}
\begin{figure}[h]
\centering
\caption{Wykres $\Delta$(n) dla K = 1.01}
\includegraphics[scale=0.33]{f5.jpg}
\end{figure}
\begin{center}
Obszar pracy $\Delta$(n) = 4.70 $\cdot$ 10$^{-3}$ - 0.76 $\cdot$ 10$^{-3}$ = 3.94 $\cdot$ 10$^{-3}$
\end{center}
\begin{figure}[h]
\centering
\caption{Wykres $\Delta$(n) dla K = 1.1}
\includegraphics[scale=0.33]{f6.jpg}
\end{figure}
\begin{center}
Obszar pracy $\Delta$(n) = 10.15 $\cdot$ 10$^{-3}$ - 0.29 $\cdot$ 10$^{-3}$ = 9.86 $\cdot$ 10$^{-3}$
\end{center}
\clearpage
\begin{figure}[h]
\centering
\caption{Wykres $\Delta$(n) dla K = 1.5}
\includegraphics[scale=0.33]{f7.jpg}
\end{figure}
\begin{center}
Obszar pracy $\Delta$(n) = 29.63 $\cdot$ 10$^{-3}$ - 0.20 $\cdot$ 10$^{-3}$ = 29.43 $\cdot$ 10$^{-3}$
\end{center}
\begin{figure}[h]
\centering
\caption{Wykres $\Delta$(n) dla K = 2.0}
\includegraphics[scale=0.33]{f8.jpg}
\end{figure}
\begin{center}
Obszar pracy $\Delta$(n) = 50.00 $\cdot$ 10$^{-3}$ - 0.19 $\cdot$ 10$^{-3}$ = 49.81 $\cdot$ 10$^{-3}$
\end{center}
\indent\indent Z wykresów na rys. 6 - 9 możemy wyznaczyć obszar pracy. Dla K $\leq$ 1.5 obszar pracy rozszerza się w górę i w dół. Dla K = 2.0 obszar pracy rośnie jedynie w górę. Dla K $\geq$ 1.5 pojawia się też szum śrutowy i staje się on bardzo silny już dla K = 2 (obserowany na rys. 9 nieustannie). Przyjmujemy więc obszar poszukiwań K w zakresie (1; 2].
\clearpage
\subsection{Wyznaczanie optymalnej wartości K}
\begin{figure}[h]
\centering
\caption{Wykres SQNR = f(K), $\Delta$(0) = 0.01}
\includegraphics[scale=0.33]{f9.jpg}
\end{figure}
\begin{figure}[h]
\centering
\caption{Porównanie sygnału oryginalnego z sygnałem modulowanym}
\includegraphics[scale=0.33]{f10.jpg}
\end{figure}
\indent\indent Dla wartości K = 1.5 otrzymujemy najlepszy SQNR. Funkcja SQNR = f(K) osiąga wtedy maksimum i przyjmuje wartość f(1.5) = 13.56 dB. Na rys. 11 widzimy, że w otoczeniu lokalnych maksimów funkcji oryginalnej występuje lekki szum śrutowy, jednak jak wskazuje SQNR jest on optymalny względem jakości sygnału.
\clearpage
\subsection{Wyznaczanie optymalnej długości filtru adaptacyjnego}
\begin{figure}[h]
\centering
\caption{Wykres SQNR = f(p)}
\includegraphics[scale=0.33]{f11.jpg}
\end{figure}
Rozważając optymalną długość filtru w sensie SQNR, najwyższą wartość otrzymujemy dla p~=~8. Jest to SQNR = 27.7 dB. To tylko o 0.15 dB więcej względem filtru o długości p = 5 i~1.18 dB względem p = 3. Są to różnice najczęściej niesłyszalne dla człowieka, warte rozważenia jest więc przyjęcie funkcji kosztu zależnej nie tylko od SQNR, ale także od ilości mnożeń wykonywanych w~ciągu sekundy: V(SQNR, R) = SQNR $\pm$ $\gamma\cdot$R. Dodatkowo dla p = 10 SQNR zaczyna maleć.
\subsection{Tabele podsumowujące odczytane z wykresów własności}
% Table generated by Excel2LaTeX from sheet 'Arkusz1'
\begin{table}[h]
  \centering
  \caption{Podsumowanie okresu przejściowego, obszaru pracy oraz uwagi dla zmiennego K}
    \begin{tabular}{|c|c|c|c|}\hline
    $K$ & $T_p$ & $\Delta(n)~\cdot 10^{-3}$ & Uwagi \\\hline
    1.01 & 600 & 3.94 & dla $\Delta$(0) = 0.01 brak stanu przejściowego \\\hline
    1.1 & 160 & 9.86 & dla $\Delta$(0) = 0.01 brak stanu przejściowego \\\hline
    1.5 & 145 & 29.43 & dla $\Delta$(0) = 0.01 brak stanu przejściowego, efekt szumu śrutowego \\\hline
    2 & 100 & 49.81 & dla $\Delta$(0) = 0.01 brak stanu przejściowego, nieustanny szum śrutowy \\\hline
    \end{tabular}%
  \label{tab:addlabel}%
\end{table}%
% Table generated by Excel2LaTeX from sheet 'Arkusz1'
\begin{table}[h]
  \centering
  \caption{Wartość SQNR, zmiana SQNR, ilość elementarnych mnożeń (K) oraz ilość mnożeń w~ciągu 1 sekundy (R) w zależności od długości filtru (p)}
    \begin{tabular}{|c|c|c|c|c|}\hline
    $P$ & $SQNR~[dB]$ & $\Delta SQNR~[dB]$ & $K$ & $R$ \\\hline
    0 & 21 & 0 & 1 & 8000 \\\hline
    3 & 26.52 & 5.52 & 10 & 80 000 \\\hline
    5 & 27.55 & 1.03 & 16 & 128 000 \\\hline
    8 & 27.7 & 0.15 & 25 & 200 000 \\\hline
    10 & 27.66 & -0.04 & 31 & 248 000 \\\hline
    \end{tabular}%
  \label{tab:addlabel}%
\end{table}%
Przy filtrze krótszym od wartości optymalnej p = 8 nie ma wyraźnego obniżenia SQNR. Tracąc jedynie 0.15 dB, dla p = 5, zysk w ilości mnożeń wynosi 72 000. Przy obniżeniu do p = 3, SQNR spada o~1.18 dB względem maksimum, lecz zysk ilości mnożeń wynosi aż 120 000.
\clearpage
\section{Wnioski}
\subsection{ADM}
\begin{itemize}
\item Pętla sprzężenia zwrotnego uzależniona jest od wartości parametru K. W miarę wzrostu K~maleje czas okresu przejściowego.
\item Zwiększając K do wartości 1.5, obserwujemy zwiększenie obszaru pracy zarówno w dół, jak i~w~górę. Dla wartości K = 2 obserwujemy jedynie wzrost obszaru pracy w górę.
\item Przyjmując dowolne z badanych K, obserwujemy że dla $\Delta$(0) = 0.01 nie ma stanu przejściowego, co jest pożądanym efektem i pozwala sprowadzić problem optymalizacji do 1D.
\item Wyznaczona optymalna wartość K = 1.5 przy $\Delta$(0) = 0.01. Dla tych parametrów osiągamy najwyższą wartość SQNR = 13.56 dB.
\item Dla K = 1.5 zaobserwowano szum śrutowy w okolicach lokalnych maksimów funkcji oryginalnej, jednak jak pokazano jest to optymalna wartość parametru K w sensie SQNR.
\end{itemize}
\subsection{ADPCM}
\begin{itemize}
\item Optymalna długość filtru p = 8 w sensie SQNR.
\item Warto również wziąć pod uwagę (jednak nie jest to element wykonanego ćwiczenia) p = 3, a~także p = 5. Pomiędzy p = 3, a p = 8 obserwujemy $\Delta$SQNR = 1.18 dB, a pomiędzy p~=~5 i~p~=~8 obserwujemy jedynie $\Delta$SQNR = 0.15 dB. Są to wartości na tyle małe, że można rozważyć skrócenie filtru bez utraty jakości.
\item Skracając filtr do p = 5 obniża się ilość mnożeń o 72 000 w ciągu sekundy.
\item Skracając filtr do p = 3 obniża się ilość mnożeń o 120 000 w ciągu sekundy.
\item Dla filtru o długości p = 10 obserwujemy spadek SQNR, czyli występuje nadmierne dopasowanie powodujące spadek jakości - overfitting.
\item Do realizacji kodera ADPCM potrzebna jest szybkość transmisji 32 kb/s, ponieważ każda próbka reprezentowana jest przez 4 bity. Pobieramy 8 000 próbek na sekundę (częstotliwość próbkowania 8 kHz), więc: 4 bity $\cdot$ 8000 $\frac{1}{s}$ = 32 000 $\frac{b}{s}$ = 32 $\frac{kb}{s}$.
\end{itemize}
\clearpage
\section{Literatura}
[1] Materiały udostępnione do wykładu dr. inż. Roberta Hossy.
\end{document}